\documentclass{article}
\usepackage[utf8]{inputenc}


\title{Configuração da Aplicação Web Livraria}
\author{Grupo 3}
\date{March 2016}

\begin{document}

\maketitle

\section{Introdução}
Este manual busca instruir o usuário a configurar o banco de dados e a aplicação, além de executar simulações de acesso e leitura de seus resultados.

\section{Configurando o SSH}

Para preparar o ambiente, é recomendável configurar o acesso à máquina do experimento por SSH.
Na máquina de onde SSH vai ser executado, deve-se editar o arquivo ~/.ssh/config, e adicionar as seguintes linhas.
\begin{verbatim}
  
Host ncbn3
HostName cbn3.lab.ic.unicamp.br
User raXXXYYY

\end{verbatim}

onde XXXYYY é o numero do RA.

Geração de Chaves:

Para acessar a máquina cbn3 do cluster sem o uso de senhas, deve-se gerar um par de chaves pública e privada.

Para isso, deve-se abrir o terminal na sua máquina e executar os seguintes comandos:
\begin{verbatim}
 ssh-keygen -t rsa
\end{verbatim}

Esse comando gera as chaves pública (id_rsa.pub) e privada (id_rsa) e salva-as na pasta $ ~/.ssh$

O comando vai pedir um nome de arquivo e uma senha/confirmação de senha. Deixe todos em branco.
\begin{verbatim}

 ssh-copy-id cbn3
\end{verbatim}

O programa pedirá a senha para logar na máquina cbn3. Neste caso, é o mesmo login e senha da rede do IC.

Esse comando copia a chave publica para o arquivo authorized_keys na pasta $.ssh$.

Com isso, o acesso por SSH está configurado. Para logar na máquina cbn3 sem senha, basta executar no terminal:

\begin{verbatim}
  ssh cbn3
\end{verbatim}

\section{Configurando o PostgreSQL}

PostgreSQL é o banco de dados utilizado pela aplicação.

Uma vez instalado o PostgreSQL 9.5, um cluster novo pode ser criado com os seguintes comandos.
\begin{verbatim}
 pg_dropcluster --stop 9.5 main

 pg_createcluster 9.5 main -u [user_name]
\end{verbatim}

Um novo usuário, chamado postgres, será criado e configurado como administrador do banco de dados. Para acessá-lo, deve-se executar:
\begin{verbatim}
 su -s /bin/bash postgres

\end{verbatim}
 senha: postgres
Como usuário postgres, pode-se acessar o banco de dados com o comando:
\begin{verbatim}
 psql -l 
\end{verbatim}

\section{Configurando o Tomcat}

Apache Tomcat é o servidor web a ser usado para o projeto.

A instalação do servidor cria um usuário $tomcat7$, para execução dos servidores, e um grupo $tomcat7$, para que usuários possam instanciar aplicações web.

Para se configurar o servidor:

Faça o download do PostgreSQL JDBC, em https://jdbc.postgresql.org/download.html, e adicione o conteudo à pasta $CATALINA\_BASE/lib$.

Abra o arquivo $CATALINA\_BASE/conf/context.xml$ e copie a seguinte linha ao bloco Context:
\begin{verbatim}
  <ResourceLink type="javax.sql.DataSource" name="jdbc/TPCW" global="jdbc/TPCWPool"/>

\end{verbatim}


Abra o arquivo $CATALINA\_BASE/conf/server.xml$ e adicione os seguintes linhas ao bloco GlobalNamingResources:

\begin{verbatim}
     <Resource type="javax.sql.DataSource"
              name="jdbc/TPCWPool"
              factory="org.apache.tomcat.jdbc.pool.DataSourceFactory"
              driverClassName="org.postgresql.Driver"
              url="jdbc:postgresql://localhost:5432/tpcw"
              username="<user_name>"
              password="<user_password>"
              initialSize="10"
              maxActive="200"
              maxIdle="100"
              minIdle="10" />
\end{verbatim}
   

Reinicie o Tomcat com o seguinte comando:
\begin{verbatim}
  # invoke-rc.d tomcat7 restart   
\end{verbatim}


\section{Configurando o TPC-W}

Para se ter permissão para criar aplicações na pasta $CATALINA\_BASE/webapps$, onde estão as configurações do Tomcat, deve-se adicionar o usuário ao grupo $tomcat7$. Utilize o seguinte comando.

\begin{verbatim}
  # adduser [usuário_comum] tomcat7
\end{verbatim}

Faça login novamente para que o novo grupo seja carregado.

Com isso, a aplicação web já pode ser carregada, acessando-se

$http://cbn3.lab.ic.unicamp.br:8080/tpcw/TPCW\_home\_interaction$

em uma máquina da rede do IC.


\section{Utilizando o Emulador de Navegadores RBE}

Execute o comando a seguir para iniciar o emulador com um acesso. Se quiser executar mais acessos simultâneos, substitua o número 1, antes de $-ITEM$, pelo valor desejado.

\begin{verbatim}
    java -cp /nfs/rbe/rbe.jar rbe.RBE -EB rbe.EBTPCW1Factory 1 -CUST
    10 -ITEM 1000 -OUT out.debug.m
    -WWW http://cbn3.lab.ic.unicamp.br:8080/tpcw/
    -RU 10 -MI 120 -RD 10 -DEBUG 10


\end{verbatim}

    Após o acesso, será gerado o arquivo de saída $out.debug.m$, com dados e estatísticas. Se quiser realizar outro test, apague este arquivo antes de executar o RBE novamente.

    Para analisar o arquivo de saída gerado pelo RBE, execute:
\begin{verbatim}
    python /nfs/rbe/analyze.py out.m
\end{verbatim}

Com isso, os gráficos e dados serão extraídos do arquivo de saída.

Para visualizar os gráficos, execute o comando:

\begin{verbatim}
    cat tpcw-run.plot
\end{verbatim}

Copie os resultados obtidos.

Abra o programa $gnuplot$, com o comando:

\begin{verbatim}
    gnuplot
\end{verbatim}

Dentro do terminal do $gnuplot$, execute:

\begin{verbatim}
    set terminal dumb
\end{verbatim}

Com isso, a visualização dos gráficos será simplificada.

Agora, cole o resultado copiado anteriormente no terminal do $gnuplot$ e execute. Os gráficos de estatísticas de acesso desejados serão disponibilizados na tela.

\end{document}
