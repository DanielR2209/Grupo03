\documentclass{article}
\usepackage[utf8]{inputenc}

\title{Aplicação TPC-W altamente disponível}
\author{Grupo 3}
\date{March 2016}


\begin{document}

\maketitle

\section{Resumo}

Este relatório tem como propósito mostrar todos os procedimentos utilizados para rodar o TPC-W na máquina cbn3 do cluster disponibilizado pelo professor Buzato.

Após isso, é mostrado as análises do desempenho dos perfis de benchmark disponíveis pelo componente RBE do TPC-W.

\section{Introdução}
O TPC-W[1] é uma ferramenta usada para o benchmark de servidores web e testar várias plataformas de hardware contra cada um para ganhar uma melhor visão dos métodos que iriam melhorar aplicações da vida real. Nós trabalhamos com o padrão implementado TPC-W por estudantes da Universidade de Wisconsin-Madison.

O benchmark funciona a grosso modo da maneira seguinte: há um servidor web rodando um serviço de e-commerce que recebe uma carga de requisições de clientes com perfis de navegação emulados. Esta caarga é medida pelo número de Emulated Browsers (EB). A performance devolvida pelo sistema também pode ser medido por WIPS(Web Interactions Per Second) ou por WIRT(Web Interaction Response Time).

Como apontado por vários estudos [2,4], a variação do tempo de resposta é uma informação muito importante. Antes de tentar entender o comportamento da latência através do sistema, nós temos que explorar a arquitetura e topologia ambos do TPC-W como da infra-estrutura que suporta ela.




\section{Gerando a aplicação Web}
O primeiro passo foi gerar a aplicação web a ser estudada.

Para isso, utilizamos os seguintes aplicativos.


Banco de dados PostgreSQL, versão 9.5
    
Servidor web Tomcat
    
Web Benchmark TPC-W
    
Emulador de navegadores RBE

Gnuplot, para gerar gráficos.


A máquina utilizada para os experimentos possui as seguintes especificações:

Sistema Operacional Ubuntu 14.04

CPU Intel(R) Core(TM)2 Quad CPU    Q8400  @ 2.66GHz

4GB de memória RAM





\section{Resultados medidos}



\end{document}
